\chapter{Introdução}
\label{ch:intro}
\section{Propósito}
    O presente projeto pretende criar um site para gerenciamento de eventos chamado \textit{CriasEvent}, para que os organizadores possam criar e gerenciar eventos, enquanto os participantes possam se inscrever nos mesmos.
\section{Audiência}
    O projeto se destina a organizadores de evento que querem uma plataforma para organizar e disponibilizar seus eventos, tanto presencias como onlines, e para os participantes que desejam se inscrever e participar de eventos, com o diferencial de poderem fazer inscrições para diferentes eventos em uma mesma página.
\section{Uso previsto}
  Espera-se que o site \textit{CriasEvent} seja acessado por usuários cadastrados, que podem ser tanto organizadores quanto participantes. 

  Para participatnes, o site será uma plataforma de inscrição e de consulta, onde o usuário Participante poderá ver quais eventos estão sendo divulgados, e se desejar, se increver nos mesmos, bem como consultar quais eventos já está inscrito. 

  Para organizadores, o site será utilizado para gerenciamento dos eventos, onde os mesmos poderão ser criados, deletados e atualizados pelo organizador, bem como disponibilizados e customizados na página de eventos



\chapter{Descrição Geral}
\label{Descrição Geral}

\section{Classe de Usuários e suas características}
\textit{CriasEvent} tem basicamente três classes de uso:
    \begin{itemize}
        \item \textbf{Organizadores}: Criam e gerenciam seus eventos.
        \item \textbf{Participantes}: Acessam, Buscam e se inscrevem nos eventos disponíveis.
        \item \textbf{Administrador}: Controlam todo o conteúdo do site(Frontend e Backend).
    \end{itemize}

\section{Escopo}
    \textit{CriasEvent} será capaz de:
    \begin{itemize}
      \item Exibição em Interface Web: todos os eventos serão exibidos através de uma página web. 
      \item Gerenciamento de eventos: CRUD dos eventos: Criação, Consulta, Atualização e Remoção de eventos.
      \item Gerenciamento de inscrições: Para participantes haverá Confirmação(testar se o participante pode se inscrever no evento), e para Organizadores haverá acesso a relatórios de inscritos e participantes.
      \item Busca e filtro de eventos: Participantes poderão buscar por eventos baseados em palavras-chave.
      \item Inscrição em eventos.
      \item Acesso às informações detalhadas dos eventos inscritos: data de ocorrência, local, requisitos, etc.
    \end{itemize}

\section{Necessidades de Usuários}
	\textit{CriasEvent} pretende criar uma plataforma simples e sucinta, permitindo aos participantes visualizarem, se increverem em eventos. Note que o site não hospeda nenhum evento em si, apenas informações de inscrição e links relevantes de participação. Para os organizdores, \textit{CriasEvent} pretende ser uma plataforma onde os mesmos possam executar operações de CRUD de uma forma acessível e simples para poderem gerenciar seus eventos.

\section{Ambiente de Operação}
	O \textit{website CriasEvent} funcionará em qualquer navegador, logo  em qualquer sistema operacional com acesso a um navegador.

\begin{tabular}{>{\raggedright}p{1.5cm}>{\raggedright}p{4cm}>{\raggedright}p{6cm}>{\raggedright}p{4cm}}
\toprule
  \textbf{N°} & \textbf{Módulo} & \textbf{Descrição} & \textbf{Tecnologias usadas} \tabularnewline 
\midrule
  1 & Página Web & Onde os usuários terão acesso as funcionalidades do escopo. & HTML/CSS/javascript\tabularnewline \hline
  2 & Banco de Dados & Onde serão armazenados todas as informações(Usuários Cadastrados, Eventos, etc) & MySQL\tabularnewline \hline
  3 & BackEnd & Permitirá a comunicação de 2 com 1 & Python(Flask)\tabularnewline
\bottomrule
\end{tabular}

\section{Restrições}
    O sucesso do projeto pode ser interrompido se um ou mais do seguintes fatores ocorrerem:
    \begin{itemize}
        \item Falha de atender aos requerimentos listados pela entidade superior.
        \item Falha de se cumprir com o prazo de entrega determindado.
        \item Incapacidade dos autores de implementarem as funcionalidades exigidas por conta de falta de conhecimento da utilização das ferramentas necessárias.
    \end{itemize}

\newpage

\chapter{Requisitos}
\label{Requisitos}

\section{Requisitos Funcionais Página Web}

\begin{tabular}{>{\raggedright}p{1.5cm}>{\raggedright}p{4cm}>{\raggedright}p{10cm}}
\toprule
\textbf{ID} & \textbf{Requirement} & \textbf{Description} \tabularnewline 
\midrule
  RF101 & Homepage & Pagina inicial exibindo os eventos disponiveis, caixa de busca e botão de cadastro.\tabularnewline \hline
  RF102 & Cadastro/Login & Pagina de Cadastro com caixas de informações para preenchimento e possibilidade de login direto.\tabularnewline \hline
  RF103 & Gerenciamento de Evento (Organizador) & Pagina exclusiva para Usuário Organizador onde será possível realizar o Crud dos eventos\tabularnewline \hline 
  RF104 & Busca de Eventos & Realizar a busca e filtragem de eventos da Homepage baseado em palavras-chave.\tabularnewline \hline
  RF105 & Inscrição Eventos & Página para inscrição no evento selecionado.\tabularnewline \hline
  RF106 & Página de Usuário & Página com as informçãoes do usuário, e.g.: Eventos incritos/gerenciados e informações de cadastro.\tabularnewline \hline
  RF107 & Página de Evento & Página com a informação do evento selecionado.\tabularnewline
\bottomrule
\end{tabular}

\section{Requisitos Funcionais Banco de Dados}
\begin{tabular}{>{\raggedright}p{1.5cm}>{\raggedright}p{4cm}>{\raggedright}p{10cm}}
\toprule
\textbf{ID} & \textbf{Requirement} & \textbf{Description} \tabularnewline 
\midrule
  RF201 & Criação de evento & Permitir ao usuário criar um evento, atribuindo as informações necessárias.\tabularnewline \hline
  RF202 & Consulta de evento & Filtrar eventos com base em informações e palavras chaves\tabularnewline \hline
  RF203 & Atualizção de evento & Adicionar, remover ou alterar informções sobre o evento.\tabularnewline \hline
  RF204 & Remoção de evento & Excluir um evento.\tabularnewline \hline
  RF205 & Gerar Lista de Presença & Exportar uma lista de inscritos no evento.\tabularnewline 
\bottomrule
\end{tabular}

\section{Requisitos Funcionais Backend}

\begin{tabular}{>{\raggedright}p{1.5cm}>{\raggedright}p{4cm}>{\raggedright}p{10cm}}
\toprule
\textbf{ID} & \textbf{Requirement} & \textbf{Description} \tabularnewline 
\midrule
RF-001 & CRUD do banco de dados dos eventos. & Utilizar mySQL para permitir criação, edição, exculsão, atualização dos eventos para os organizadores \tabularnewline \hline
RF-002 & Criação da página Web & utilzar html, css e javascript para criar a página web onde os usuários poderam se cadastrar, se inscrever em eventos,  caso sejam participantes, e para os organizadores, permitir o uso das funções implementados no banco de dados pelo RF-001 \tabularnewline \hline
RF-003 & Busca e filtro de eventos & permitir aos participantes realizar buscas no banco de dados por palavras-chave e filtrar eventos(e.g. por local, tipo de evento, artistas/palestrantes, período, etc) bem como não exibir eventos aos participantes que não podem se increver nos mesmos através condições de participação(e.g. ser maior de 18 anos). \tabularnewline \hline
RF-004 & Descrição detalhada de cada evento & Para eventos presenciais, o sistema deve disponibilizar todas as informações necessárias para que os participantes possam comparecer ao evento. ara eventos on-line, o sistema deve disponibilizar os links de acesso ao evento por meio de comunicações enviadas aos inscritos. \tabularnewline \hline
RF-005 & Criação de formulários de inscrição personalizados. & Cada evento precisa de um formulário de incrição, que não é o mesmo para todos. \tabularnewline \hline
RF-006 & Gerenciamento de inscrições & Os organizadores precisam de informações sobre seus eventos, como confirmação de inscrição, lista de presença etc. \tabularnewline
\bottomrule
\end{tabular}


\section{Requisitos não funcionais}
\begin{tabular}{>{\raggedright}p{1.5cm}>{\raggedright}p{4cm}>{\raggedright}p{10cm}}
\toprule
\textbf{ID} & \textbf{Requirement} & \textbf{Description} \tabularnewline 
\midrule
RNF-001 & Sistema de notificações & Organizadores poderão se comunicar com os inscritos por e-mail ou mensagens no sistema, participantes receberão notificações sobre novos eventos, alterações em eventos inscritos etc. \tabularnewline \hline
RNF-002 & Perfis para organizadores e participantes. & Os perfis devem mostrar informações sobre seu usuário, como Histórico de eventos participados e/ou organizados. \tabularnewline \hline
RNF-003 & Configurações de notificações e privacidade. & Permitir aos usuário configurar quando serem notificados e quais dados seram compartilhados \tabularnewline \hline
RNF-004 & Sistema Multimidia & Os organizadores e participantes poderam fazer Upload de fotos e vídeos, promocionais ou não. \tabularnewline \hline 
RNF-005 & Comentários sobre as fotos. & Permitir que os usuários se comuniquem através de comentário nas fotos e vídeos postados. \tabularnewline \hline 
RNF-006 & Moderação de fotos e comentários. & Permitir aos adiminstradores de eventos realizarem moderação sobre os conteúdos postados baseados em suas \textit{guidelines} \tabularnewline \hline
RNF-007 & Integração com Redes socias & Permitir o Compartilhamento de eventos nas redes sociais. \tabularnewline \hline
RNF-008 & Visualização de eventos no maps. & Integrar a localização dos eventos ao google maps, mostrando seus locais.\tabularnewline

\bottomrule
\end{tabular}

\section{Definição de Risco}

\begin{appendices}
\chapter{Glossário}
\begin{itemize}
 \item CriasEvent: Nome do sistema de gerenciamento de eventos que está sendo desenvolvido. O objetivo principal é permitir que organizadores criem e gerenciem eventos e que participantes possam se inscrever neles

\item Organizadores: Classe de usuários do CriasEvent que têm a capacidade de criar e gerenciar eventos na plataforma Eles podem executar operações de CRUD (criar, ler, atualizar e excluir) nos dados dos eventos.

\item Participantes: Classe de usuários do CriasEvent que podem buscar e se inscrever nos eventos disponíveis.

\item Administrador: Classe de usuários do CriasEvent que têm controle sobre todo o conteúdo do site.

\item Evento: Uma ocorrência, tanto presencial quanto online, que pode ser criada e gerenciada por organizadores através do sistema CriasEvent. O sistema permite o preenchimento de informações, upload de imagens e definição de vagas para eventos.

\item Inscrição: O processo pelo qual os participantes demonstram interesse em participar de um evento através do sistema CriasEvent . O sistema também oferece gerenciamento de inscrições, incluindo confirmação e lista de presença 

\item CRUD: Acrônimo para as operações fundamentais de Criar, Read (Ler), Update (Atualizar) e Delete (Excluir) dados no banco de dados. No contexto do CriasEvent, os organizadores utilizarão essas operações para gerenciar os eventos . O banco de dados para essas operações será o MySQL.

\item Requisitos Funcionais: Descrevem as funcionalidades que o sistema CriasEvent deverá implementar, ou seja, o que o sistema deverá ser capaz de fazer. Exemplos incluem o CRUD de eventos, a criação da página web, a busca e filtro de eventos e o gerenciamento de inscrições.

\item Requisitos Não Funcionais: Descrevem as qualidades ou restrições do sistema CriasEvent, como desempenho, usabilidade, segurança, etc. Exemplos incluem o sistema de notificações, perfis para organizadores e participantes e a integração com redes sociais.
\end{itemize}
\end{appendices}


